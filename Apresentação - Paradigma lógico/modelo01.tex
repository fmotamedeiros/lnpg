%%%%%%%%%%%%%%%%%%%%%%%%%%%%%%%%%%%%%%%%%
% Beamer Presentation
% LaTeX Template
% Version 1.0 (10/11/12)
%
% This template has been downloaded from:
% http://www.LaTeXTemplates.com
%
% License:
% CC BY-NC-SA 3.0 (http://creativecommons.org/licenses/by-nc-sa/3.0/)
%
%%%%%%%%%%%%%%%%%%%%%%%%%%%%%%%%%%%%%%%%%

%----------------------------------------------------------------------------------------
%   PACKAGES AND THEMES
%----------------------------------------------------------------------------------------

\documentclass{beamer}

\mode<presentation> {

% The Beamer class comes with a number of default slide themes
% which change the colors and layouts of slides. Below this is a list
% of all the themes, uncomment each in turn to see what they look like.

%\usetheme{default}
%\usetheme{AnnArbor}
%\usetheme{Antibes}
%\usetheme{Bergen}
%\usetheme{Berkeley}
%\usetheme{Berlin}
%\usetheme{Boadilla}
%\usetheme{CambridgeUS}
%\usetheme{Copenhagen}
%\usetheme{Darmstadt}
%\usetheme{Dresden}
%\usetheme{Frankfurt}
%\usetheme{Goettingen}
%\usetheme{Hannover}
%\usetheme{Ilmenau}
%\usetheme{JuanLesPins}
%\usetheme{Luebeck}
\usetheme{Madrid}
%\usetheme{Malmoe}
%\usetheme{Marburg}
%\usetheme{Montpellier}
%\usetheme{PaloAlto}
%\usetheme{Pittsburgh}
%\usetheme{Rochester}
%\usetheme{Singapore}
%\usetheme{Szeged}
%\usetheme{Warsaw}

% As well as themes, the Beamer class has a number of color themes
% for any slide theme. Uncomment each of these in turn to see how it
% changes the colors of your current slide theme.

%\usecolortheme{albatross}
%\usecolortheme{beaver}
%\usecolortheme{beetle}
%\usecolortheme{crane}
%\usecolortheme{dolphin}
%\usecolortheme{dove}
%\usecolortheme{fly}
%\usecolortheme{lily}
%\usecolortheme{orchid}
%\usecolortheme{rose}
%\usecolortheme{seagull}
%\usecolortheme{seahorse}
%\usecolortheme{whale}
%\usecolortheme{wolverine}

%\setbeamertemplate{footline} % To remove the footer line in all slides uncomment this line
%\setbeamertemplate{footline}[page number] % To replace the footer line in all slides with a simple slide count uncomment this line

%\setbeamertemplate{navigation symbols}{} % To remove the navigation symbols from the bottom of all slides uncomment this line
}

\usepackage{graphicx} % Allows including images
\usepackage{booktabs} % Allows the use of \toprule, \midrule and \bottomrule in tables
\usepackage[brazilian]{babel}
\usepackage[utf8]{inputenc}
\usepackage[T1]{fontenc}

%----------------------------------------------------------------------------------------
%   TITLE PAGE
%----------------------------------------------------------------------------------------

\title[Introdução ao paradigma Lógico]{Paradigma Lógico} % The short title appears at the bottom of every slide, the full title is only on the title page
\author{Jonatan Lessa dos Santos \\
        José Carlos Gomes Filho \\
        Laís Francelly Omena Silva \\
        Leonardo Henrique dos Anjos Santos} % Student Name 1

\institute[IFAL] % Your institution as it will appear on the bottom of every slide, may be shorthand to save space
{
Instituto Federal de Alagoas \\ % Your institution for the title page
\medskip
\textit{Semestre 2018-2} % Your email address
}
\date{\today} % Date, can be changed to a custom date

\begin{document}

\begin{frame}
\titlepage % Print the title page as the first slide
\end{frame}

%----------------------------------------------------------------------------------------
%   PRESENTATION SLIDES
%----------------------------------------------------------------------------------------

%------------------------------------------------
\section{First Section} % Sections can be created in order to organize your presentation into discrete blocks, all sections and subsections are automatically printed in the table of contents as an overview of the talk
%------------------------------------------------

\subsection{Subsection Example} % A subsection can be created just before a set of slides with a common theme to further break down your presentation into chunks

\begin{frame}
\frametitle{História do Paradigma Lógico}
\begin{itemize}
    \item John McCarthy
    \item Década de 70 
    \item Planner, Prolog, Qlisp, etc
\end{itemize}


\end{frame}

%------------------------------------------------

\begin{frame}
\frametitle{O paradigma Lógico}
\begin{itemize}
\item Forma Declarativa
\item O que são fatos?
\item Ex: O sol é uma estrela.
\item O que são regras?
\item Ex: A Terra é um planeta se ela não for uma estrela.
\item O que envolve a Programação Lógica?
\end{itemize}
\end{frame}

%------------------------------------------------

\begin{frame}
\frametitle{Prolog}
\begin{itemize}
    \item O termo prolog é oriundo de 'programming in logic' ou simplesmente programação lógica
    \item Ex: 'O sol é uma estrela' == estrela(sol)
    \item Ex:  'A Terra é um planeta se
ela não for uma estrela' ==  planeta(terra) :- not(estrela(terra))
    \item Ex: 'A terra é um planeta?' == ?- planeta(terra)
    \item 'O homem é um mamífero e todo mamífero é um animal. O homem é um animal?' == mamífero(homem), animal(mamífero), ?-animal(homem).
\end{itemize}


\end{frame}

%------------------------------------------------
\begin{frame}
\frametitle{Aplicações}
\begin{itemize}
    \item IA 
    \item Sistemas Especialistas
    \item Processamento de Linguagem Natural
    \item Prova de Teoremas
\end{itemize}


\end{frame}


%------------------------------------------------

\begin{frame}
\Huge{\centerline{Referências}}
\begin{itemize}
    \item Paradigma Lógico: Prof Sergio Souza Costa
    \item Learn Prolog: lpn.swi-prolog.org
\end{itemize}
\end{frame}

%----------------------------------------------------------------------------------------

\end{document}